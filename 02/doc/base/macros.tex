% **************************************************************************************************
% ** SPSC Report and Thesis Template
% **************************************************************************************************
%
% ***** Authors *****
% Daniel Arnitz, Paul Meissner, Stefan Petrik
% Signal Processing and Speech Communication Laboratory (SPSC)
% Graz University of Technology (TU Graz), Austria
%
% ***** Changelog *****
%
% ***** Todo *****
%
% **************************************************************************************************



% **************************************************************************************************
% * SECTIONING AND TEXT
% **************************************************************************************************

% new chapter, section, ... plus a few addons
%   part
\newcommand{\newpart}[2]{\FloatBarrier\cleardoublepage\part{#1}\label{part:#2}}%
%   chapter
\newcommand{\newchapter}[2]{\FloatBarrier\chapter{#1}\label{chp:#2}}
\newcommand{\newchapterNoTOC}[2]{\FloatBarrier\stepcounter{chapter}\chapter*{#1}\label{chp:#2}}%
%   section
\newcommand{\newsection}[2]{\FloatBarrier\vspace{5mm}\section{#1}\label{sec:#2}}%
\newcommand{\newsectionNoTOC}[2]{\FloatBarrier\vspace{5mm}\stepcounter{section}\section*{#1}\label{sec:#2}}%
%   subsection
\newcommand{\newsubsection}[2]{\FloatBarrier\vspace{3mm}\subsection{#1}\label{sec:#2}}%
\newcommand{\newsubsectionNoTOC}[2]{\FloatBarrier\vspace{3mm}\stepcounter{subsection}\subsection*{#1}\label{sec:#2}}%
%   subsubsection
\newcommand{\newsubsubsection}[2]{\vspace{2mm}\subsubsection{#1}\label{sec:#2}}%
\newcommand{\newsubsubsectionNoTOC}[2]{\vspace{2mm}\stepcounter{subsubsection}\subsubsection*{#1}\label{sec:#2}}%

% next paragraph
\newcommand{\nxtpar}{\par\bigskip}

% "stylish" quotes on the right side
\newcommand{\openingquote}[2]{\hfill\parbox[t]{10cm}{\itshape\raggedleft{"#1"}\\\footnotesize -- #2}\nxtpar}%

% direct quotes
% \newenvironment{directquote}{\nxtpar\hrule}{\hrule}\hfill\litref{#1}{#2}}

% warnings and attention signs in marginpar
\newcommand{\MDanger}{\marginpar{\Huge\centering\fbox{\textbf{!}}}}%
\newcommand{\MAttention}{\marginpar{\Huge\centering\textbf{!}}}%
\newcommand{\MHint}{\marginpar{\Huge\centering\textbf{\checkmark}}}%
\newcommand{\MQuestion}{\marginpar{\Huge\centering\textbf{?}}}%

% same footnote number as last one
\newcommand{\lastfootnotemark}{\addtocounter{footnote}{-1}\footnotemark}%

% value-unit commands (for 457 kHz, etc)
\newcommand{\vu}[2]{\mbox{$#1\,\text{#2}$}} % "value~unit" ... prevents e.g. 456 \linebreak mV
\newcommand{\vuc}[3]{\mbox{$#1\,\text{#2}\;#3\,\%$}} % "value~unit~tolerance-per-cent"
\newcommand{\vum}[3]{\mbox{$#1\,\text{#2}\;#3\,\perthousand$}} % "value~unit~tolerance-per-mil"

% reminders
\newcommand{\reminder}[1]{\colorbox{red}{#1}\xspace}%
\newcommand{\rem}{\reminder{(...)}}%
\newcommand{\remq}{\reminder{???}}%
\newcommand{\uc}{\nxtpar\colorbox{yellow}{... under construction ...}\nxtpar}%

% misc
\newcommand{\pwd}{.} % present working directory (can be used to create relativ paths per part, etc.)


% **************************************************************************************************
% * MATH
% **************************************************************************************************

% highlighting
\newcommand{\vm}[1]{\bm{#1}}% vector or matrix

% operators
\newcommand{\E}[1]{\text{E}\!\left\{#1\right\}}% expectation operator
\newcommand{\var}[1]{\text{var}\!\left\{#1\right\}}% variance operator
\renewcommand{\ln}[1]{\text{ln}\!\left(#1\right)}% natural logarithm
\newcommand{\ld}[1]{\text{ld}\!\left(#1\right)}% logarithm base 2
\renewcommand{\log}[1]{\text{log}\!\left(#1\right)}% logarithm (base 10)
\newcommand{\logb}[2]{\text{log}_{#1}\!\left(#2\right)}% logarithm base ...
\newcommand{\avgvar}[1]{\overline{\text{var}}\!\left\{#1\right\}}% average variance operator
\renewcommand{\Re}[1]{\text{Re}\!\left\{#1\right\}}% real part
\renewcommand{\Im}[1]{\text{Im}\!\left\{#1\right\}}% imaginary part

% other
\newcommand{\conj}{^\ast}% conjugate complex
\newcommand{\mtx}[2]{\left[\begin{array}{#1}#2\end{array}\right]}%vector/matrix


% **************************************************************************************************
% * FLOATS (FIGURES, TABLES, LISTINGS, ...)
% **************************************************************************************************

% figures without frames
%   standard
\newcommand{\fig}[3]{\begin{figure}\centering\includegraphics[width=\textwidth]{#1}\caption{#2}\label{fig:#3}\end{figure}}%
%   with controllable parameters
\newcommand{\figc}[4]{\begin{figure}\centering\includegraphics[#1]{#2}\caption{#3}\label{fig:#4}\end{figure}}%
%   two subfigures
\newcommand{\twofig}[6]{\begin{figure}\centering%
\subfigure[#2]{\includegraphics[width=0.495\textwidth]{#1}}%
\subfigure[#4]{\includegraphics[width=0.495\textwidth]{#3}}%
\caption{#5}\label{fig:#6}\end{figure}}%
%   two subfigures and controllable parameters
\newcommand{\twofigc}[8]{\begin{figure}\centering%
\subfigure[#3]{\includegraphics[#1]{#2}}%
\subfigure[#6]{\includegraphics[#4]{#5}}%
\caption{#7}\label{fig:#8}\end{figure}}%

% framed figures
%   standard
\newcommand{\figf}[3]{\begin{figure}\centering\fbox{\includegraphics[width=\textwidth]{#1}}\caption{#2}\label{fig:#3}\end{figure}}%
%   with controllable parameters
\newcommand{\figcf}[4]{\begin{figure}\centering\fbox{\includegraphics[#1]{#2}}\caption{#3}\label{fig:#4}\end{figure}}%
%   two subfigures
\newcommand{\twofigf}[6]{\begin{figure}\centering%
\fbox{\subfigure[#2]{\includegraphics[width=0.495\textwidth]{#1}}}%
\fbox{\subfigure[#4]{\includegraphics[width=0.495\textwidth]{#3}}}%
\caption{#5}\label{fig:#6}\end{figure}}%
%   two subfigures and controllable parameters
\newcommand{\twofigcf}[8]{\begin{figure}\centering%
\fbox{\subfigure[#3]{\includegraphics[#1]{#2}}}%
\fbox{\subfigure[#6]{\includegraphics[#4]{#5}}}%
\caption{#7}\label{fig:#8}\end{figure}}%

% listings
\newcommand{\filelisting}[4]{\lstinputlisting[print=true,language=#1,caption={#3},label={lst:#4}]{#2}}

% preserve backslash for linebreaks in tables (ragged... redefines \\, thus it has to be preserved)
\newcommand{\pbs}[1]{\let\temp=\\#1\let\\=\temp}%
