\subsection{Pose Recognition}
Nachdem das 2 layer feed-forward neural network mit 6 hidden units und 300 Epochen trainiert wurde, konnte
eine Classifizierungsrate von 100\% erreicht werden.

Hier die Confusion Matrix, wobei man erkennt, dass es sich bei der Matrix um eine Diagonalmatrix handelt. Das
bedeutet, dass alle Bilder korrekt klassifiziert wurden.
\begin{verbatim}
 confusion_matrix =
   141     0     0     0
     0   141     0     0
     0     0   138     0
     0     0     0   144
\end{verbatim}

In Abbildung \ref{fig:pose_weights} wurden die Gewichte der hidden neurons dargestellt. Ein hellerer Pixelwert
entspricht dabei einem größerem Gewichtswert.
Mit der Außnahme von Neuron 3 konzentrieren sich die Bereiche mit den größeren Gewichten vorüberwiegend
auf den Bereich in dem sich das Gesicht befindet. Bei Neuron 3 sind viele Gewichte im Bereich des Hintergrunds
relativ hell. Die verschiedenen Neuronen dürften sich also auf unterschiedliche Bereiche des Gesichts konzentrieren.
\begin{figure}[hp!]
\begin{center}
 \includegraphics[width=0.6\textwidth]{./figures/3_1_weights}
 \caption{Plots of the weights of the hidden neurons}
\label{fig:pose_weights}
\end{center}
\end{figure}




