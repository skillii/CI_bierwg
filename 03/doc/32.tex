\subsection{Face Recognition}

Im Histogramm über dem MSE erkennt man, dass der MSE stark von den Anfangsgewichten abhängt.
Wie zu erwarten war der MSE bei den Testdaten etwas größer als bei den Trainingsdaten.




\begin{figure}[hp!]
\begin{center}
 \includegraphics[width=0.5\textwidth]{./figures/3_2_hist_test}
 \caption{Histogramm vom MSE über den Testdaten}
\label{fig:face_false_wrong}
\end{center}
\end{figure}

\begin{figure}[hp!]
\begin{center}
 \includegraphics[width=0.5\textwidth]{./figures/3_2_hist_train}
 \caption{Histogramm vom MSE über den Trainingsdaten}
\label{fig:face_false_wrong}
\end{center}
\end{figure}



\begin{figure}[hp!]
\begin{center}
 \includegraphics[width=0.99\textwidth]{./figures/3_2_false-wrong_classification}
 \caption{Example Images for some correct and some wrong classified Images}
\label{fig:face_false_wrong}
\end{center}
\end{figure}

In der Abbildung \ref{fig:face_false_wrong} wurden einige falsch und korrekt klassifizierte Bilder dargestellt.
Faktoren, die die Klassifikationsperformance verschlechtern, sind unserer Meinung Artefakte im Hintergrund, wie 
es Beispielsweise in Abbildung \ref{fig:face_false_wrong} Spalte 3 Zeile 2 zu erkennen ist.
Bei den mittleren Bildern in der ersten Zeile ist unserer Meinung die schräge Blickrichtung Schuld an der falschen
Klassifizierung. Zwei falsch klassifizierte Bilder waren auch deutlich dunkler als die anderen Bilder. Ein weiterer
Einflussfaktor für eine falsche Klassifizierung ist auch, wenn sich der Kopf der Person nicht genau im Mittelpunkt
des Bildes befindet.


